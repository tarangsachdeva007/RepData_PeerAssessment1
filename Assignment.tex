% Options for packages loaded elsewhere
\PassOptionsToPackage{unicode}{hyperref}
\PassOptionsToPackage{hyphens}{url}
%
\documentclass[
]{article}
\usepackage{lmodern}
\usepackage{amssymb,amsmath}
\usepackage{ifxetex,ifluatex}
\ifnum 0\ifxetex 1\fi\ifluatex 1\fi=0 % if pdftex
  \usepackage[T1]{fontenc}
  \usepackage[utf8]{inputenc}
  \usepackage{textcomp} % provide euro and other symbols
\else % if luatex or xetex
  \usepackage{unicode-math}
  \defaultfontfeatures{Scale=MatchLowercase}
  \defaultfontfeatures[\rmfamily]{Ligatures=TeX,Scale=1}
\fi
% Use upquote if available, for straight quotes in verbatim environments
\IfFileExists{upquote.sty}{\usepackage{upquote}}{}
\IfFileExists{microtype.sty}{% use microtype if available
  \usepackage[]{microtype}
  \UseMicrotypeSet[protrusion]{basicmath} % disable protrusion for tt fonts
}{}
\makeatletter
\@ifundefined{KOMAClassName}{% if non-KOMA class
  \IfFileExists{parskip.sty}{%
    \usepackage{parskip}
  }{% else
    \setlength{\parindent}{0pt}
    \setlength{\parskip}{6pt plus 2pt minus 1pt}}
}{% if KOMA class
  \KOMAoptions{parskip=half}}
\makeatother
\usepackage{xcolor}
\IfFileExists{xurl.sty}{\usepackage{xurl}}{} % add URL line breaks if available
\IfFileExists{bookmark.sty}{\usepackage{bookmark}}{\usepackage{hyperref}}
\hypersetup{
  pdftitle={Assignment},
  hidelinks,
  pdfcreator={LaTeX via pandoc}}
\urlstyle{same} % disable monospaced font for URLs
\usepackage[margin=1in]{geometry}
\usepackage{color}
\usepackage{fancyvrb}
\newcommand{\VerbBar}{|}
\newcommand{\VERB}{\Verb[commandchars=\\\{\}]}
\DefineVerbatimEnvironment{Highlighting}{Verbatim}{commandchars=\\\{\}}
% Add ',fontsize=\small' for more characters per line
\usepackage{framed}
\definecolor{shadecolor}{RGB}{248,248,248}
\newenvironment{Shaded}{\begin{snugshade}}{\end{snugshade}}
\newcommand{\AlertTok}[1]{\textcolor[rgb]{0.94,0.16,0.16}{#1}}
\newcommand{\AnnotationTok}[1]{\textcolor[rgb]{0.56,0.35,0.01}{\textbf{\textit{#1}}}}
\newcommand{\AttributeTok}[1]{\textcolor[rgb]{0.77,0.63,0.00}{#1}}
\newcommand{\BaseNTok}[1]{\textcolor[rgb]{0.00,0.00,0.81}{#1}}
\newcommand{\BuiltInTok}[1]{#1}
\newcommand{\CharTok}[1]{\textcolor[rgb]{0.31,0.60,0.02}{#1}}
\newcommand{\CommentTok}[1]{\textcolor[rgb]{0.56,0.35,0.01}{\textit{#1}}}
\newcommand{\CommentVarTok}[1]{\textcolor[rgb]{0.56,0.35,0.01}{\textbf{\textit{#1}}}}
\newcommand{\ConstantTok}[1]{\textcolor[rgb]{0.00,0.00,0.00}{#1}}
\newcommand{\ControlFlowTok}[1]{\textcolor[rgb]{0.13,0.29,0.53}{\textbf{#1}}}
\newcommand{\DataTypeTok}[1]{\textcolor[rgb]{0.13,0.29,0.53}{#1}}
\newcommand{\DecValTok}[1]{\textcolor[rgb]{0.00,0.00,0.81}{#1}}
\newcommand{\DocumentationTok}[1]{\textcolor[rgb]{0.56,0.35,0.01}{\textbf{\textit{#1}}}}
\newcommand{\ErrorTok}[1]{\textcolor[rgb]{0.64,0.00,0.00}{\textbf{#1}}}
\newcommand{\ExtensionTok}[1]{#1}
\newcommand{\FloatTok}[1]{\textcolor[rgb]{0.00,0.00,0.81}{#1}}
\newcommand{\FunctionTok}[1]{\textcolor[rgb]{0.00,0.00,0.00}{#1}}
\newcommand{\ImportTok}[1]{#1}
\newcommand{\InformationTok}[1]{\textcolor[rgb]{0.56,0.35,0.01}{\textbf{\textit{#1}}}}
\newcommand{\KeywordTok}[1]{\textcolor[rgb]{0.13,0.29,0.53}{\textbf{#1}}}
\newcommand{\NormalTok}[1]{#1}
\newcommand{\OperatorTok}[1]{\textcolor[rgb]{0.81,0.36,0.00}{\textbf{#1}}}
\newcommand{\OtherTok}[1]{\textcolor[rgb]{0.56,0.35,0.01}{#1}}
\newcommand{\PreprocessorTok}[1]{\textcolor[rgb]{0.56,0.35,0.01}{\textit{#1}}}
\newcommand{\RegionMarkerTok}[1]{#1}
\newcommand{\SpecialCharTok}[1]{\textcolor[rgb]{0.00,0.00,0.00}{#1}}
\newcommand{\SpecialStringTok}[1]{\textcolor[rgb]{0.31,0.60,0.02}{#1}}
\newcommand{\StringTok}[1]{\textcolor[rgb]{0.31,0.60,0.02}{#1}}
\newcommand{\VariableTok}[1]{\textcolor[rgb]{0.00,0.00,0.00}{#1}}
\newcommand{\VerbatimStringTok}[1]{\textcolor[rgb]{0.31,0.60,0.02}{#1}}
\newcommand{\WarningTok}[1]{\textcolor[rgb]{0.56,0.35,0.01}{\textbf{\textit{#1}}}}
\usepackage{graphicx,grffile}
\makeatletter
\def\maxwidth{\ifdim\Gin@nat@width>\linewidth\linewidth\else\Gin@nat@width\fi}
\def\maxheight{\ifdim\Gin@nat@height>\textheight\textheight\else\Gin@nat@height\fi}
\makeatother
% Scale images if necessary, so that they will not overflow the page
% margins by default, and it is still possible to overwrite the defaults
% using explicit options in \includegraphics[width, height, ...]{}
\setkeys{Gin}{width=\maxwidth,height=\maxheight,keepaspectratio}
% Set default figure placement to htbp
\makeatletter
\def\fps@figure{htbp}
\makeatother
\setlength{\emergencystretch}{3em} % prevent overfull lines
\providecommand{\tightlist}{%
  \setlength{\itemsep}{0pt}\setlength{\parskip}{0pt}}
\setcounter{secnumdepth}{-\maxdimen} % remove section numbering

\title{Assignment}
\author{}
\date{\vspace{-2.5em}}

\begin{document}
\maketitle

\#\#Reading the data and Remove the NA values and store in a separate
structure for future use

\hypertarget{calculate-the-total-number-of-steps-taken-per-day}{%
\subsection{Calculate the total number of steps taken per
day}\label{calculate-the-total-number-of-steps-taken-per-day}}

\begin{Shaded}
\begin{Highlighting}[]
\NormalTok{steps_per_day <-}\StringTok{ }\KeywordTok{aggregate}\NormalTok{(steps }\OperatorTok{~}\StringTok{ }\NormalTok{date, good_act, sum)}
\end{Highlighting}
\end{Shaded}

\hypertarget{create-a-histogram-of-no-of-steps-per-day}{%
\subsection{Create a histogram of no of steps per
day}\label{create-a-histogram-of-no-of-steps-per-day}}

\begin{Shaded}
\begin{Highlighting}[]
\KeywordTok{hist}\NormalTok{(steps_per_day}\OperatorTok{$}\NormalTok{steps, }\DataTypeTok{main =} \StringTok{"Histogram of total number of steps per day"}\NormalTok{, }\DataTypeTok{xlab =} \StringTok{"Steps per day"}\NormalTok{)}
\end{Highlighting}
\end{Shaded}

\includegraphics{Assignment_files/figure-latex/unnamed-chunk-2-1.pdf}

\hypertarget{calculate-the-mean-and-median-of-the-total-number-of-steps-taken-per-day}{%
\section{Calculate the mean and median of the total number of steps
taken per
day}\label{calculate-the-mean-and-median-of-the-total-number-of-steps-taken-per-day}}

\begin{Shaded}
\begin{Highlighting}[]
\KeywordTok{round}\NormalTok{(}\KeywordTok{mean}\NormalTok{(steps_per_day}\OperatorTok{$}\NormalTok{steps))}
\end{Highlighting}
\end{Shaded}

\begin{verbatim}
## [1] 10766
\end{verbatim}

\begin{Shaded}
\begin{Highlighting}[]
\KeywordTok{median}\NormalTok{(steps_per_day}\OperatorTok{$}\NormalTok{steps)}
\end{Highlighting}
\end{Shaded}

\begin{verbatim}
## [1] 10765
\end{verbatim}

\hypertarget{calculate-average-steps-per-interval-for-all-days}{%
\section{Calculate average steps per interval for all
days}\label{calculate-average-steps-per-interval-for-all-days}}

\begin{Shaded}
\begin{Highlighting}[]
\NormalTok{avg_steps_per_interval <-}\StringTok{ }\KeywordTok{aggregate}\NormalTok{(steps }\OperatorTok{~}\StringTok{ }\NormalTok{interval, good_act, mean)}
\end{Highlighting}
\end{Shaded}

\hypertarget{calculate-average-steps-per-day-for-all-intervals---not-required-but-for-my-own-sake}{%
\section{Calculate average steps per day for all intervals - Not
required, but for my own
sake}\label{calculate-average-steps-per-day-for-all-intervals---not-required-but-for-my-own-sake}}

\begin{Shaded}
\begin{Highlighting}[]
\NormalTok{avg_steps_per_day <-}\StringTok{ }\KeywordTok{aggregate}\NormalTok{(steps }\OperatorTok{~}\StringTok{ }\NormalTok{date, good_act, mean)}
\end{Highlighting}
\end{Shaded}

\hypertarget{plot-the-time-series-with-appropriate-labels-and-heading}{%
\section{Plot the time series with appropriate labels and
heading}\label{plot-the-time-series-with-appropriate-labels-and-heading}}

\begin{Shaded}
\begin{Highlighting}[]
\KeywordTok{plot}\NormalTok{(avg_steps_per_interval}\OperatorTok{$}\NormalTok{interval, avg_steps_per_interval}\OperatorTok{$}\NormalTok{steps, }\DataTypeTok{type=}\StringTok{'l'}\NormalTok{, }\DataTypeTok{col=}\DecValTok{1}\NormalTok{, }\DataTypeTok{main=}\StringTok{"Average number of steps by Interval"}\NormalTok{, }\DataTypeTok{xlab=}\StringTok{"Time Intervals"}\NormalTok{, }\DataTypeTok{ylab=}\StringTok{"Average number of steps"}\NormalTok{)}
\end{Highlighting}
\end{Shaded}

\includegraphics{Assignment_files/figure-latex/unnamed-chunk-6-1.pdf}

\hypertarget{identify-the-interval-index-which-has-the-highest-average-steps}{%
\section{Identify the interval index which has the highest average
steps}\label{identify-the-interval-index-which-has-the-highest-average-steps}}

\begin{Shaded}
\begin{Highlighting}[]
\NormalTok{interval_idx <-}\StringTok{ }\KeywordTok{which.max}\NormalTok{(avg_steps_per_interval}\OperatorTok{$}\NormalTok{steps)}
\end{Highlighting}
\end{Shaded}

\hypertarget{identify-the-specific-interval-and-the-average-steps-for-that-interval}{%
\section{Identify the specific interval and the average steps for that
interval}\label{identify-the-specific-interval-and-the-average-steps-for-that-interval}}

\begin{Shaded}
\begin{Highlighting}[]
\KeywordTok{print}\NormalTok{ (}\KeywordTok{paste}\NormalTok{(}\StringTok{"The interval with the highest avg steps is "}\NormalTok{, avg_steps_per_interval[interval_idx, ]}\OperatorTok{$}\NormalTok{interval, }\StringTok{" and the no of steps for that interval is "}\NormalTok{, }\KeywordTok{round}\NormalTok{(avg_steps_per_interval[interval_idx, ]}\OperatorTok{$}\NormalTok{steps, }\DataTypeTok{digits =} \DecValTok{1}\NormalTok{)))}
\end{Highlighting}
\end{Shaded}

\begin{verbatim}
## [1] "The interval with the highest avg steps is  835  and the no of steps for that interval is  206.2"
\end{verbatim}

\hypertarget{calculate-the-number-of-rows-with-missing-values}{%
\section{Calculate the number of rows with missing
values}\label{calculate-the-number-of-rows-with-missing-values}}

\begin{Shaded}
\begin{Highlighting}[]
\NormalTok{missing_value_act <-}\StringTok{ }\NormalTok{activity[}\OperatorTok{!}\KeywordTok{complete.cases}\NormalTok{(activity), ]}
\KeywordTok{nrow}\NormalTok{(missing_value_act)}
\end{Highlighting}
\end{Shaded}

\begin{verbatim}
## [1] 2304
\end{verbatim}

\hypertarget{loop-thru-all-the-rows-of-activity-find-the-one-with-na-for-steps.}{%
\section{Loop thru all the rows of activity, find the one with NA for
steps.}\label{loop-thru-all-the-rows-of-activity-find-the-one-with-na-for-steps.}}

\hypertarget{for-each-identify-the-interval-for-that-row}{%
\section{For each identify the interval for that
row}\label{for-each-identify-the-interval-for-that-row}}

\hypertarget{then-identify-the-avg-steps-for-that-interval-in-avg_steps_per_interval}{%
\section{Then identify the avg steps for that interval in
avg\_steps\_per\_interval}\label{then-identify-the-avg-steps-for-that-interval-in-avg_steps_per_interval}}

\hypertarget{substitute-the-na-value-with-that-value}{%
\section{Substitute the NA value with that
value}\label{substitute-the-na-value-with-that-value}}

\begin{Shaded}
\begin{Highlighting}[]
\ControlFlowTok{for}\NormalTok{ (i }\ControlFlowTok{in} \DecValTok{1}\OperatorTok{:}\KeywordTok{nrow}\NormalTok{(activity)) \{}
  \ControlFlowTok{if}\NormalTok{(}\KeywordTok{is.na}\NormalTok{(activity}\OperatorTok{$}\NormalTok{steps[i])) \{}
\NormalTok{    val <-}\StringTok{ }\NormalTok{avg_steps_per_interval}\OperatorTok{$}\NormalTok{steps[}\KeywordTok{which}\NormalTok{(avg_steps_per_interval}\OperatorTok{$}\NormalTok{interval }\OperatorTok{==}\StringTok{ }\NormalTok{activity}\OperatorTok{$}\NormalTok{interval[i])]}
\NormalTok{    activity}\OperatorTok{$}\NormalTok{steps[i] <-}\StringTok{ }\NormalTok{val }
\NormalTok{  \}}
\NormalTok{\}}
\end{Highlighting}
\end{Shaded}

\hypertarget{aggregate-the-steps-per-day-with-the-imputed-values}{%
\section{Aggregate the steps per day with the imputed
values}\label{aggregate-the-steps-per-day-with-the-imputed-values}}

\begin{Shaded}
\begin{Highlighting}[]
\NormalTok{steps_per_day_impute <-}\StringTok{ }\KeywordTok{aggregate}\NormalTok{(steps }\OperatorTok{~}\StringTok{ }\NormalTok{date, activity, sum)}
\end{Highlighting}
\end{Shaded}

\hypertarget{draw-a-histogram-of-the-value}{%
\section{Draw a histogram of the
value}\label{draw-a-histogram-of-the-value}}

\begin{Shaded}
\begin{Highlighting}[]
\KeywordTok{hist}\NormalTok{(steps_per_day_impute}\OperatorTok{$}\NormalTok{steps, }\DataTypeTok{main =} \StringTok{"Histogram of total number of steps per day (IMPUTED)"}\NormalTok{, }\DataTypeTok{xlab =} \StringTok{"Steps per day"}\NormalTok{)}
\end{Highlighting}
\end{Shaded}

\includegraphics{Assignment_files/figure-latex/unnamed-chunk-12-1.pdf}

\hypertarget{compute-the-mean-and-median-of-the-imputed-value}{%
\section{Compute the mean and median of the imputed
value}\label{compute-the-mean-and-median-of-the-imputed-value}}

\hypertarget{calculate-the-mean-and-median-of-the-total-number-of-steps-taken-per-day-1}{%
\section{Calculate the mean and median of the total number of steps
taken per
day}\label{calculate-the-mean-and-median-of-the-total-number-of-steps-taken-per-day-1}}

\begin{Shaded}
\begin{Highlighting}[]
\KeywordTok{round}\NormalTok{(}\KeywordTok{mean}\NormalTok{(steps_per_day_impute}\OperatorTok{$}\NormalTok{steps))}
\end{Highlighting}
\end{Shaded}

\begin{verbatim}
## [1] 10766
\end{verbatim}

\begin{Shaded}
\begin{Highlighting}[]
\KeywordTok{median}\NormalTok{(steps_per_day_impute}\OperatorTok{$}\NormalTok{steps)}
\end{Highlighting}
\end{Shaded}

\begin{verbatim}
## [1] 10766.19
\end{verbatim}

\begin{Shaded}
\begin{Highlighting}[]
\NormalTok{week_day <-}\StringTok{ }\ControlFlowTok{function}\NormalTok{(date_val) \{}
\NormalTok{  wd <-}\StringTok{ }\KeywordTok{weekdays}\NormalTok{(}\KeywordTok{as.Date}\NormalTok{(date_val, }\StringTok{'%Y-%m-%d'}\NormalTok{))}
  \ControlFlowTok{if}\NormalTok{  (}\OperatorTok{!}\NormalTok{(wd }\OperatorTok{==}\StringTok{ 'Saturday'} \OperatorTok{||}\StringTok{ }\NormalTok{wd }\OperatorTok{==}\StringTok{ 'Sunday'}\NormalTok{)) \{}
\NormalTok{    x <-}\StringTok{ 'Weekday'}
\NormalTok{  \} }\ControlFlowTok{else}\NormalTok{ \{}
\NormalTok{    x <-}\StringTok{ 'Weekend'}
\NormalTok{  \}}
\NormalTok{  x}
\NormalTok{\}}
\end{Highlighting}
\end{Shaded}

\hypertarget{apply-the-week_day-function-and-add-a-new-column-to-activity-dataset}{%
\section{Apply the week\_day function and add a new column to activity
dataset}\label{apply-the-week_day-function-and-add-a-new-column-to-activity-dataset}}

\begin{Shaded}
\begin{Highlighting}[]
\NormalTok{activity}\OperatorTok{$}\NormalTok{day_type <-}\StringTok{ }\KeywordTok{as.factor}\NormalTok{(}\KeywordTok{sapply}\NormalTok{(activity}\OperatorTok{$}\NormalTok{date, week_day))}
\end{Highlighting}
\end{Shaded}

\#load the ggplot library and Create the aggregated data frame by
intervals and day\_type

\begin{Shaded}
\begin{Highlighting}[]
\KeywordTok{library}\NormalTok{(ggplot2)}
\end{Highlighting}
\end{Shaded}

\begin{verbatim}
## Warning: package 'ggplot2' was built under R version 4.0.2
\end{verbatim}

\begin{Shaded}
\begin{Highlighting}[]
\NormalTok{steps_per_day_impute <-}\StringTok{ }\KeywordTok{aggregate}\NormalTok{(steps }\OperatorTok{~}\StringTok{ }\NormalTok{interval}\OperatorTok{+}\NormalTok{day_type, activity, mean)}
\end{Highlighting}
\end{Shaded}

\hypertarget{create-the-plot}{%
\section{Create the plot}\label{create-the-plot}}

\begin{Shaded}
\begin{Highlighting}[]
\NormalTok{plt <-}\StringTok{ }\KeywordTok{ggplot}\NormalTok{(steps_per_day_impute, }\KeywordTok{aes}\NormalTok{(interval, steps)) }\OperatorTok{+}
\StringTok{  }\KeywordTok{geom_line}\NormalTok{(}\DataTypeTok{stat =} \StringTok{"identity"}\NormalTok{, }\KeywordTok{aes}\NormalTok{(}\DataTypeTok{colour =}\NormalTok{ day_type)) }\OperatorTok{+}
\StringTok{  }\KeywordTok{theme_gray}\NormalTok{() }\OperatorTok{+}
\StringTok{  }\KeywordTok{facet_grid}\NormalTok{(day_type }\OperatorTok{~}\StringTok{ }\NormalTok{., }\DataTypeTok{scales=}\StringTok{"fixed"}\NormalTok{, }\DataTypeTok{space=}\StringTok{"fixed"}\NormalTok{) }\OperatorTok{+}
\StringTok{  }\KeywordTok{labs}\NormalTok{(}\DataTypeTok{x=}\StringTok{"Interval"}\NormalTok{, }\DataTypeTok{y=}\KeywordTok{expression}\NormalTok{(}\StringTok{"No of Steps"}\NormalTok{)) }\OperatorTok{+}
\StringTok{  }\KeywordTok{ggtitle}\NormalTok{(}\StringTok{"No of steps Per Interval by day type"}\NormalTok{)}
\KeywordTok{print}\NormalTok{(plt)}
\end{Highlighting}
\end{Shaded}

\includegraphics{Assignment_files/figure-latex/unnamed-chunk-16-1.pdf}

\end{document}
